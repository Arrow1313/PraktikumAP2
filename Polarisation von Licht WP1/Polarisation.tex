\documentclass[12pt]{scrartcl}

 

\usepackage[utf8]{inputenc}

\usepackage[T1]{fontenc}

\usepackage{lmodern}

\usepackage[ngerman]{babel}

\usepackage{amsmath}

\usepackage{graphicx}


 

\title{Versuch WP1\\ Polarisation von Licht}

\author{Frederik Strothmann, Henrik Jürgens}

\date{\today}


\begin{document}


 %deckblatt erstellen

\maketitle
\tableofcontents
\newpage

%einleitung zu dem experiment

\section{Einleitung}
Mit einer Photozelle wird untersucht, wie sich Licht verhält, wenn es durch Polarisationsfilter tritt (Gesetz von Malus), an einem trüben Medium (Tröpfchen-Suspension in Wasser) gestreut wird oder an einer Glasplatte reflektiert wird (Brewsterscher Winkel).
Danach werden die Eigenschaften von zirkular oder elliptisch polarisierten Lichtwellen untersucht sowie ihre Wechselwirkung mit Materie (Cellophan, Zuckerlösung zur Demonstration der optischen Aktivität, Metallspiegel). 

%versuchsaufbau mit skizze

\section{Versuchsaufbau}


\section{Versuchsdurchführung}


\subsection{Praktische Durchführung}


\subsection{Theoretische Durchführung}


\section{Messergebnisse}



\section{Auswertung}


\section{Diskussion}


 %Werte stimmen mit den Formeln überein/nicht überein

\end{document}


\documentclass[12pt]{scrartcl}

 

\usepackage[utf8]{inputenc}

\usepackage[T1]{fontenc}

\usepackage{lmodern}

\usepackage[ngerman]{babel}

\usepackage{amsmath}

\usepackage{graphicx}

\usepackage{float}


 

\title{Versuch WP1\\ Polarisation von Licht}

\author{Frederik Strothmann, Henrik Jürgens}

\date{\today}


\begin{document}


 %deckblatt erstellen

\maketitle
\tableofcontents
\newpage

%einleitung zu dem experiment

\section{Einleitung}
Mit einer Photozelle untersuchen wir, wie sich Licht verhält, wenn es durch Polarisationsfilter tritt (Gesetz von Malus), an einem trüben Medium (Tröpfchen-Suspension in Wasser) gestreut wird oder an einer Glasplatte reflektiert wird (Brewsterscher Winkel).
Danach werden die Eigenschaften von zirkular oder elliptisch polarisierten Lichtwellen untersucht sowie ihre Wechselwirkung mit Materie (Cellophan, Zuckerlösung zur Demonstration der optischen Aktivität, Metallspiegel). 

%versuchsaufbau mit skizze

\section{Versuchsaufbau}

\section{Versuch WO1.1: Verifizierung des Malusschen Gesetzes}
\subsection{Versuchsdurchführung}
\subsubsection{Praktische Durchführung}
Wir bauen auf einer optischen Bank die in 
der folgenden Abbildung skizzierte Anordnung auf.
%Bitte Abbildung 4 aus der Versuchsbeschreibung einfügen
Wir verwenden eine Glühlampe mit Kondensor, der so eingestellt ist, daß das von der Glühlampe ausgehende Licht auf die Photozelle fokussiert ist. Eine Irisblende zwischen Kondensor und Polarisator dient der Regelung der Lichtintensität. Die
Lichintensität wird so eingestellt, dass der Photostrom $I_A$ der Photozelle 
%(siehe Abb. 3 und Gleichungen 21, 22)
den Wert von 1,0$\mu$A nicht überschreitet. ($U_A \approx$ 70V, $R_A$ = 1M$\Omega$) Der Photostrom $I_A$ wird anschließend für verschiedene Winkel $\theta$ bestimmt um das Malussche Gesetz zu überprüfen.
Dafür stellen wir $I_A$ als Funktion von $\theta$ grafisch dar.
Zuletzt stellen wir einen dritten Polarisator zwischen die beiden anderen, deren Durchlaßrichtung
um 90$^\circ$ gegeneinander verdreht ist. Wir wollen den Winkel, bei dem der größte Anteil des Lichtes durchgelassen wird, bestimmen.
%Erklären Sie schriftlich Ihre Beobachtung.
\subsubsection{Theoretische Durchführung}
Nach dem Malusschen Gesetz ist der folgende Zusammenhang zu erwarten:
\begin{align}
I_A \propto <S> = \frac{1}{2} \varepsilon_0 c E'^2
\end{align}
$I_A$ der Anodenstrom, $S$ die Energieflussdichte und $E'$ die Amplitude der Lichtwelle hinter dem Polarisator.
Für $E'$ gilt:
\begin{align}
E' = E_0 \cos{\theta}
\end{align}
$E_0$ die Anfangsamplitude des Lichtes.\\
Schließlich Folgt:
\begin{align}
I_A \propto \cos^2{\theta}
\end{align}
\subsection{Diskussion}

\section{Versuch WO1.2: Beobachtung der Polarisation von Licht durch Einfachstreuung}
\subsection{Versuchsdurchführung}
\subsubsection{Praktische Durchführung}
Wir bauen auf einer optischen Bank den in der Folgenden Abbildung skizzierten
Versuch auf.
%Bitte Abbildung 10 aus der Versuchsbeschreibung einfügen
Als Lichtquelle dient eine Glühlampe mit Kondensor. Das von dieser Lichtquelle ausgehende Lichtbündel fällt auf eine wässrige Lösung von Styrofan in einer rechteckigen Glasküvette.
Wir sehen uns das unter
90$^\circ$ gestreute Licht durch einen Polarisationsfilter (= Analysator) an. 
%Was beobachten Sie bei Drehung des Analysators? Ist das Streulicht polarisiert? Wenn ja, in welche Richtung? Ist die Polarisation vollständig?
\subsubsection{Theoretische Durchführung}
\subsection{Diskussion}

\section{Versuch WO1.3:
Messung der Richtungscharakteristik der Strahlung einer schwingenden Ladung}
\subsection{Versuchsdurchführung}
\subsubsection{Praktische Durchführung}
Wir verwenden den Versuchsaufbau zu Versuch WO1.2, um den in der folgenden Abbildung skizzierten Aufbau zu realisieren (lediglich der Polarisationsfilter wechselt seinen Platz).
%Bitte Abbildung 11 aus der Versuchsbeschreibung einfügen
Wir drehen den Polarisationsfilter anschließend und beobachten den Intensitätsunterschied.
\subsubsection{Theoretische Durchführung}
\subsection{Diskussion}

\section{Versuch WO1.4:
Das Brewstersche Gesetz}
\subsection{Versuchsdurchführung}
\subsubsection{Praktische Durchführung}
Wir verifizieren das sogenannte Brewstersche Gesetz experimentell, indem wir im Ver-
suchsaufbau zu WO1.2 die Glasküvette durch einen Glasblock (Plexiglas) ersetzen und
die Polarisation des reflektierten Lichtes beobachten.
Der Glasblock wird dabei auf einen Drehtisch mit Winkelskala gesetzt.
%Auf dem Tisch ist eine Schiene, an die der Block angelegt wird, damit eine feste Beziehung zwischen Drehtischskala und Glasoberfläche entsteht. Die Winkelskala kann leicht verdreht sein. (Können Sie diesen Fehler kompensieren, wenn Sie den Brewsterwinkel nach beiden Seiten hin messen?) Aus den Betrachtungen zu den Versuchen WO1.2 und WO1.3 können Sie die Gültigkeit des Brewsterschen Gesetzes verstehen. 
Bei dieser Messung ist zu beachten, dass der reflektierte Strahl von ”schwingenden Glasmolekülen“ ausgesandt wird, welche in einer Ebene
senkrecht zur Ausbreitungsrichtung des gebrochenen Strahls schwingen.
Ziel ist es den Winkel zu bestimmen, bei dem das Licht nahezu vollständig polarisiert ist.(Brewsterwinkel)
%Am sichersten ist es zu beobachten, wann nahezu kein reflektiertes Licht mehr durch den (um 90$^\circ$verdrehten) Analysator gelangt.
Aus dem Brewsterwinkel wird dann %nach Gleichung \ref{}
der Brechungsindex berechnet.
\subsubsection{Theoretische Durchführung}
Für den Brewsterwinkel gilt die Beziehung:
\begin{align}
\tan{\theta} = n_G
\end{align}
$n_G$ ist der Brechungsindex von Plexiglas.\\
Der Fehler für den Brewsterwinkel ist:
\begin{align}
\sigma_\theta = (1+\tan^2{n_G})\sigma_{n_G}
\end{align}
\subsection{Diskussion}

\section{Versuch WO2.1:
Doppelbrechung von Cellophan}
\subsection{Versuchsdurchführung}
\subsubsection{Praktische Durchführung}
\begin{enumerate}
\item[a)]
Wir stellen aus einer Dreh- oder Spannhalterung und einem Stück Cellophanfolie (Klebefolie, Verpackungsfolie) ein Phasenverschiebungsplättchen (PVP) her. %(nur 1 Folie!).
Die Wirkung des PVP, die wir bei Drehung des PVP zwischen zwei gekreuzten Polaroidfiltern beoachten, soll untersucht und die Lage der optischen Achsen bestimmt werden.
%(in gleicher Halterung wie die Polarisationsfilter). Bitte verwechseln sie die Plättchen nicht mit den Filtern! Sie können die $\frac{\lamda}{4}$-Plättchen an zwei Merkmalen von den Polarisationsfiltern unterscheiden: 1) Polarisationsfilter lassen nur die Hälfte des natürlichen Lichtes durch und sind daher dunkelgrau, $\frac{\lambda}{4}$-Plättchen hingegen klar transparent. 2) Die Fassung der $\frac{\lambda}{4}$-Plättchen trägt den Aufdruck ”$\frac{\lambda}{4}$“. Die Polarisationsfilter sind unbeschriftet. 3) Keine Rolle spielt die Farbe der Drehhebel oder Aufschrift (weiß oder gelb)!
\item[b)] Wir Nehmen nun ein industrielles
$\frac{\lambda}{4}$-Plättchen und drehen es so, dass der Polarisator genau zwischen den beiden optischen Achsen (45$^\circ$) des $\frac{\lambda}{4}$-Plättchens steht.
Wir variieren die Durchlassrichtung des Analysators und beobachten Intensitäts- und Farbeffekte für das weiße Glühlampenlicht und für die fünf
Farbfilter (violett, blau, grün, gelb und rot). Die Monochromfilter 
%(achten Sie auf die fünfstellige Nummer am Filterrahmen und nicht nur auf die Filterfarbe!)
haben folgende Durchlässigkeitsbereiche (DLB) für die Wellenlänge:
%bitte Abbildung aus der Versuchsanleitung einfügen

%Für welche Farbe weist Ihr$\frac{\lambda}{4}$-Plättchen annähernd die Eigenschaften eines idealen $\frac{\lambda}{4}$-Plättchens auf ?
\item[c)] Wir haben elliptisch polarisiertes Licht hergestellt und wollen untersuchen, ob sich die Lage der Ellipse ändert, wenn wir das $\frac{\lambda}{4}$-Plättchen um 90$^\circ$
drehen. Später vergleichen wir unsere Beobachtung mit der theoretischen Erwartung.
\item[d)] Nach den Ergebnissen aus b) können wir für einen Farbfilter mit Hilfe des $\frac{\lambda}{4}$-Plättchens nahezu einen Zirkularpolarisator (ZP) herstellen, und zwar sowohl links- als auch rechtsdrehend. Wir stellen zwei ZP her und beobachten die Intensitätsverteilung bei verschiedenen Winkelstellungen zwischen den Durchlassrichtungen der beiden Polaroidfilter und den optischen Achsen der $\frac{\lambda}{4}$-Plättchen.
\end{enumerate}
\subsubsection{Theoretische Durchführung}
\begin{enumerate}
\item[a)]
Die optischen Achsen sind die Winkel, bei denen kein Licht durch den Aufbau gelangt, da das Licht einerseits nicht durch das PVP polarisiert wird und andererseits die beiden Polarisator um 90$^\circ$ zueinander verdreht sind.
\item[b)]
Wir erwarten für eine Farbe zirkular polarisiertes Licht.
Die Phasenverschiebung $\phi$ lässt sich berechnen durch:
\begin{align}
\phi = cos^{-1}\left(\frac{I_{max}-I_{min}}{I_{max}+I_{min}}\right)
\end{align}
Dabei können anstelle von $I$ auch zu $I$ proportionale Größen wie Strom oder Spannung verwendet werden.
Der Fehler ist:
\begin{align}
\sigma_{\phi} = \sqrt{
\left(\frac{\frac{2I_{max}}{(I_{max}+I_{min})^2}}{\sqrt{1-\left(\frac{I_{max}-I_{min}}{I_{max}+
I_{min}}\right)^2}}\sigma_{I_{min}}\right)^2+
\left(\frac{\frac{2I_{min}}{(I_{max}+I_{min})^2}}{\sqrt{1-\left(\frac{I_{max}-I_{min}}{I_{max}+
I_{min}}\right)^2}}\sigma_{I_{max}}\right)^2}
\end{align}
\item[c)]
Wir erwarten, dass sich die Polarisationsrichtung der Ellipse ändert, wobei die Lage der Ellipse gleich bleibt.
\item[d)]
Da wir beide ZP ($\frac{\lambda}{4}$-Plättchen um 90$^\circ$ verschoben) hintereinander geschaltet haben erwarten wir linear polarisiertes Licht.
\end{enumerate}
\subsection{Diskussion}

\section{Versuch WO2.2:
Bestimmung der spezifischen Drehung einer Zuckerlösung}
\subsection{Versuchsdurchführung}
\subsubsection{Praktische Durchführung}
\begin{enumerate}
\item[a)] Wir stellen eine Zuckerlösung bekannter Konzentration q (in [$\frac{\text{g}}{\text{cm}^3}]$)
her (bis zu etwa q = 0,5$\frac{\text{g}}{\text{cm}^3}$
sind je nach Zuckersorte und Temperatur möglich).
Diese Lösung wird in einer Glasküvette zwischen zwei Polaroidfilter gestellt und die spezifische Drehung [$\alpha$]
als Funktion der Wellenlänge des Lichts bestimmt.
\item[b)] Anschließend stellen wir eine weitere Zuckerlösung mit einer anderen Konzentration als
in a) her. Mithilfe der in a) gemessenen Werte für
[$\alpha$] bestimmen wir die Konzentration dieser zweiten Zuckerlösung und vergleichen Sie mit der aus Zucker- und
Wassergewicht errechneten.
\end{enumerate}
\subsubsection{Theoretische Durchführung}
\begin{enumerate}
\item[a)] Für die spezifische Drehung [$\alpha$] gilt die Beziehung:
\begin{align}
\alpha = [\alpha] q d
\end{align}
$\alpha$ der Winkel der Polarisation des austretenden Lichtes, q die Konzentration und d die Dicke der Zuckerlösung.
Mit dem Fehler:
\begin{align}
\sigma_{[\alpha]} = \sqrt{
\left(\frac{1}{q d}\sigma_{\alpha}\right)^2+
\left(\frac{\alpha}{q^2 d}\sigma_q\right)^2+
\left(\frac{\alpha}{q d^2}\sigma_d\right)^2}
\end{align}
\item[b)]
Die Konzentration der Zuckerlösung kann aus der spezifischen Drehung [$\alpha$], der Dicke $d$ und dem Drehwinkel $\alpha$ nach folgender Formel bestimmt werden:
\begin{align}
q = \frac{\alpha}{d [\alpha]}
\end{align}
Mit dem Fehler:
\begin{align}
\sigma_q = \sqrt{
\left(\frac{1}{d [\alpha]}\sigma_\alpha \right)^2+
\left(\frac{\alpha}{d^2 [\alpha]}\sigma_d \right)^2+
\left(\frac{\alpha}{d [\alpha]^2}\sigma_{[\alpha]}\right)^2}
\end{align}
\end{enumerate}
\subsection{Diskussion}

\section{Versuch WO2.3:
Reflexion von linear polarisiertem Licht an einer Metalloberfläche}
\subsection{Versuchsdurchführung}
\subsubsection{Praktische Durchführung}
Wir haben in allen bisherigen Versuchen im Praktikum immer nur die Wechselwirkung
von Lichtwellen mit Isolatoren (Glas, Plexiglas, Kunststoff, Zuckerlösung etc.) betrach-
tet. Mit der Vorstellung von Atomen als schwingungsfähige elektrische Dipole und mithilfe der Maxwellschen Theorie waren wir in der Lage, die beobachteten Phänomene zu erklären. Diese Vorstellungen versagen jedoch, wenn wir elektromagnetische Wellen mit Wellenlängen kleiner als 10 $\mu$m und ihre Wechselwirkung mit Metallen, also Leitern, untersuchen wollen. Wollen wir also die Reflexion von Lichtwellen an Metalloberflächen untersuchen, so versagen unsere bislang erworbenen Kenntnisse. Ein gutes Verständnis der Theorie der Leitfähigkeit von Metallen im Rahmen der Festkörperphysik, die sich wiederum auf die Quantenphysik stützt, ist hierfür notwendig. Dies soll uns jedoch nicht hindern, einige experimentelle Beobachtungen über die Reflexion von linear polarisiertem Licht an einer Metalloberfläche (Oberflächenspiegel) zu machen. Wir benötigen für die Versuche neben Lichtquelle und Oberflächenspiegel ein
$\frac{\lambda}{4}$-Plättchen und zwei Polaroidfilter und
führen damit folgende Experimente aus:
Beobachtung 1: Licht, das parallel oder senkrecht zur Einfallsebene linear polarisiert ist, ändert bei der Reflexion an einem Metallspiegel seine Polarisation nicht.
Beobachtung 2: Licht, das in einem Winkel von 45$^\circ$ zur Einfallsebene linear polarisiert ist, wird durch die Reflexion in elliptisch polarisiertes Licht umgewandelt.
Beobachtung 3: Es gibt einen Einfallswinkel, bei dem
”unter 45$^\circ$
linear polarisiertes
Licht“ als nahezu zirkular polarisiert reflektiert wird.
Eine plausible Erklärung für diese Beobachtungen finden Sie im Berkeley Kurs Band 3, Kap. 8.6, Heimversuch 26.

\section{Messergebnisse}


\section{Auswertung}


\section{Fazit}


 %Werte stimmen mit den Formeln überein/nicht überein

\end{document}


\documentclass[12pt]{scrartcl}

 

\usepackage[utf8]{inputenc}

\usepackage[T1]{fontenc}

\usepackage{lmodern}

\usepackage[ngerman]{babel}

\usepackage{amsmath}

\usepackage{graphicx}


 

\title{Versuch E45\\ RCL und RC Schwingkreis}

\author{Frederik Strothmann, Henrik Jürgens}

\date{\today}


\begin{document}


 %deckblatt erstellen

\maketitle
\tableofcontents
\newpage

%einleitung zu dem experiment

\section{Einleitung}

In diesem Versuch sollen die Eigenschaften einfacher elektrischer Stromkreise, die aus ”passiven“ Elementen wie
Widerständen, Kondensatoren und Spulen aufgebaut sind, untersucht werden. Weiterhin soll das RC-Glied und das
RL-Glied als Filter für Frequenzen benutzt werden. Je nach Schaltung entsteht ein Hoch– bzw. Tiefpass. Schließlich sollen die gedämpfte elektrische Schwingung bei einem RCL-Kreis und die erzwungenen Schwingung
in einem LC-Schwingkreis mit Hilfe des Oszillographen untersucht werden. Bei den Untersuchungen werden kommerzielle elektrische Geräte wie Funktionsgenerator, Digitalmultimeter und ein Oszillograph eingesetzt, so dass am Ende der Versuche eine gewisse Vertrautheit mit der Bedienung der Geräte
erwartet wird. Die Versuche werden mit dem Leybold-Stecksystem aufgebaut. Im Einzelnen gliedern sich die Experimente in folgende Teile:
\newline
1)
Messungen an Spannungsteilern
\newline
2)
Messung der Entladekurve eines Kondensators mit Hilfe des Oszillographen
\newline
3)
Messung der Eigenschaften eines RC- oder RL-Filters (Hoch-, Tief- und Bandpass)
\newline
4)
Untersuchung der ungedämpften und der gedämpften elektrischen Schwingung bei einem RCL-Kreis mit Hilfe des Oszillographen
%versuchsaufbau mit skizze

\section{Versuchsaufbau}


\section{Versuchsdurchführung}


\subsection{Praktische Durchführung}


\subsection{Theoretische Durchführung}


\section{Messergebnisse}



\section{Auswertung}


\section{Diskussion}


 %Werte stimmen mit den Formeln überein/nicht überein

\end{document}


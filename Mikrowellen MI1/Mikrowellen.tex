\documentclass[12pt]{scrartcl}

 

\usepackage[utf8]{inputenc}

\usepackage[T1]{fontenc}

\usepackage{lmodern}

\usepackage[ngerman]{babel}

\usepackage{amsmath}

\usepackage{graphicx}


 

\title{Versuch MI1\\ Mikrowellen}

\author{Frederik Strothmann, Henrik Jürgens}

\date{\today}


\begin{document}


 %deckblatt erstellen

\maketitle
\tableofcontents
\newpage

%einleitung zu dem experiment

\section{Einleitung}

Ein System aus Mikrowellensender und verschiedenen Empfängern ermöglicht Untersuchungen verschiedener physikalischer Effekte an Mikrowellen. So sollen in diesem Versuch stehende Wellen vermessen werden, außerdem wird die Wirkung einer Wachs-Sammellinse oder eines Polfilters (parallele Metallstäbe) untersucht sowie die Reflexion
an einer Wachsplatte (Brewster-Winkel), die Totalreflexion zwischen einer Wachs-Luft-
Wachs-Schicht und die Drehung der Polarisationsebene durch
”optisch aktive“ Substanzen
(Spiralfedern in einem Styroporträger)
%versuchsaufbau mit skizze

\section{Versuchsaufbau}


\section{Versuchsdurchführung}


\subsection{Praktische Durchführung}


\subsection{Theoretische Durchführung}


\section{Messergebnisse}



\section{Auswertung}


\section{Diskussion}


 %Werte stimmen mit den Formeln überein/nicht überein

\end{document}


\documentclass[12pt]{scrartcl}

 

\usepackage[utf8]{inputenc}

\usepackage[T1]{fontenc}

\usepackage{lmodern}

\usepackage[ngerman]{babel}

\usepackage{amsmath}

\usepackage{graphicx}


 

\title{Versuch E1\\ Drehspulinstrument}

\author{Frederik Strothmann, Henrik Jürgens}

\date{\today}


\begin{document}


 %deckblatt erstellen

\maketitle
\tableofcontents
\newpage


%einleitung zu dem experiment

\section{Einleitung}

Ein einfaches Drehspulinstrument soll durch Wahl geeigneter Vor- und Parallelwiderstände
zur Messung von Strömen und Spannungen verwendet werden.
Die Elektromotorische Kraft (EMK) und der Innenwiderstand sind wichtige Kenngrößen einer Spannungsquelle.

Im Rahmen dieses Versuchs sollen diese beiden Größen einer Batterie ausgemessen werden.
Nichtlineare Widerstandselemente spielen eine wichtige Rolle auf dem Gebiet der Elek-
tronik. Ausgemessen werden soll die Strom–Spannungs–Charakteristik einer Glühlampe
und einer Halbleiterdiode.
%versuchsaufbau mit skizze

\section{Versuchsaufbau}


\section{Versuchsdurchführung}


\subsection{Praktische Durchführung}


\subsection{Theoretische Durchführung}


\section{Messergebnisse}



\section{Auswertung}


\section{Diskussion}


 %Werte stimmen mit den Formeln überein/nicht überein

\end{document}


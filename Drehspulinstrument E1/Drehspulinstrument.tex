\documentclass[12pt]{scrartcl}

 

\usepackage[utf8]{inputenc}

\usepackage[T1]{fontenc}

\usepackage{lmodern}

\usepackage[ngerman]{babel}

\usepackage{amsmath}

\usepackage{graphicx}


 

\title{Versuch E1\\ Drehspulinstrument}

\author{Frederik Strothmann, Henrik Jürgens}

\date{\today}


\begin{document}


 %deckblatt erstellen

\maketitle
\tableofcontents
\newpage


%einleitung zu dem experiment

\section{Einleitung}

Ein einfaches Drehspulinstrument soll durch Wahl geeigneter Vor- und Parallelwiderstände
zur Messung von Strömen und Spannungen verwendet werden.
Die Elektromotorische Kraft (EMK) und der Innenwiderstand sind wichtige Kenngrößen einer Spannungsquelle.

Im Rahmen dieses Versuchs sollen diese beiden Größen einer Batterie ausgemessen werden.
Nichtlineare Widerstandselemente spielen eine wichtige Rolle auf dem Gebiet der Elek-
tronik. Ausgemessen werden soll die Strom–Spannungs–Charakteristik einer Glühlampe
und einer Halbleiterdiode.
%versuchsaufbau mit skizze

\section{Versuchsaufbau}


\section{Versuchsdurchführung}


\subsection{Praktische Durchführung}

\begin{enumerate}

	\item 
	Berechnen Sie den Innenwiderstand 			eines Drehspulmeßgerätes. Das Instrument
	zeigt Vollausschlag bei einem Strom von 		1 mA bzw. einer Spannung von 200 mV.
	\item
	Bauen Sie aus diesem Instrument mit 			Hilfe der Ihnen zur Verfügung stehenden 		Widerstände $\text{R}_1$ bis 				$\text{R}_5$ ein 							Strommeßgerät mit Vollausschlag bei 300 		mA. Wie groß ist sein Innenwiderstand?
	\item
	Das soeben gebaute Strom-Meßinstrument 		soll bei der Bestimmung des unbekannten 		Widerstands $\text{R}_x$ benutzt werden. 	Der Widerstand soll durch Strom- und 		Spannungsmessung mit Hilfe von zwei 			Schaltungen bestimmt werden.
	
	%Schaltskizzen einfügen!!!
	
	Berücksichtigen Sie bei 	der Bestimmung 		von $\text{R}_x$ die Innenwiderstände 		der in der Schaltung vorhandenen 			Meßinstrumente. Fertigen Sie zur 			Berechnung jeweils ein Ersatzschaltbild 		an. Zur Spannungsmessung benutzen Sie 		das Digitalmultimeter (DMM) mit einem 		Innenwiderstand von 10 M$\Omega$. Als 		Spannungsquelle steht Ihnen ein 				Netzgerät zur Verfügung. Benutzen Sie 		einen der variablen Ausgänge. Stellen 		Sie zunächst -- bevor Sie die 				Schaltungen 	zusammenstecken -- mit Hilfe 	des DMM die Ausgangsspannung von ca. 		1 V ein. Der Knopf für die 					Strombegrenzung ist auf Maximum zu 			stellen. Achtung: Während Sie 	eine 		Schaltung zusammenstecken, muß das 			Netzgerät ausgeschaltet sein. Erst nach 		Fertigstellung der Schaltung darf das 		Netzgerät eingeschaltet werden 				(Meßinstrumente hierbei beobachten!)
	\item
	Bauen Sie nun mit dem Drehspulinstrument 	und den Widerständen $\text{R}_1$ bis 		$\text{R}_5$ ein Spannungsmeßgerät mit 		Vollausschlag bei 1,75 V. Wie groß ist 		sein Innenwiderstand? Verwenden Sie 			dieses Gerät zur Messung der Spannung 		der Batterie. Messen Sie dieselbe 			Spannungmit dem DMM und dem 					Vielzweckmeßgerät (Unigor). Den 				Innenwiderstand des Unigors entnehmen 		Sie einer Tabelle auf der Rückseite des 		Meßgerätes. Erklären Sie die 				unterschiedlichen Meßergebnisse.
	Achtung: Die Batterie sollte nur durch 		kurzen Druck auf den Taster belastet 		werden. Warum?
	\item
	Messen Sie den Innenwiderstand der 			Batterie (Schaltschema unten!). Er wird 		größer sein als bei einer normalen 			Batterie, da ein ”zusätzlicher 				Innenwiderstand“ eingebaut ist. 				Verwenden Sie zur Spannungsmessung das 		Vielzweckinstrument (Unigor). Messen Sie
	U und I und tragen Sie U gegen I auf. 		Bestimmen Sie $\text{U}_0$ und 				$\text{R}_i$.
	
	%Schaltskizzen einfügen!!!
	
	\item
	Messen Sie die Strom-Spannungs-				Charakteristik (Kennlinie) einer 			Glühlampe und einer Diode. Schaltschema 		siehe folgende Seite. Stellen Sie die 		Ergebnisse graphisch dar. Verwenden Sie 		zur Spannungsmessung das DMM und zur 		Strommessung das Unigor. Als 				Spannungsquelle benutzen Sie den unteren 	Ausgang des Netzgerätes mit der festen 		Ausgangsspannung von 5 V (Schaltschema 		umseitig, es werden jedoch auch andere 		Netzteile mit nur einem regelbaren 			Ausgang verwendet).
	a) zur Messung der Kennlinie der 			Glühlampe
	
	%Schaltskizzen einfügen!!!
	
	b) zur Messung der Kennlinie einer Diode
	
	%Schaltskizzen einfügen!!!
	
	Was bewirkt der Vorwiderstand
	$\text{R}_v? (\text{R}_v= 16\Omega)$
\end{enumerate}


\subsection{Theoretische Durchführung}

\begin{enumerate}

	\item
	Um den Innenwiderstand des 					Drehspulmeßgerätes zu bestimmen benutzen 	wir die Formel:
	\begin{align}
	\text{R}_{i}=
	\end{align}
	\item
	\item
	\item
	\item
	\item
	
\end{enumerate}


\section{Messergebnisse}



\section{Auswertung}


\section{Diskussion}


 %Werte stimmen mit den Formeln überein/nicht überein

\end{document}


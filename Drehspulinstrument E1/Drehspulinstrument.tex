\documentclass[12pt]{scrartcl}

 

\usepackage[utf8]{inputenc}

\usepackage[T1]{fontenc}

\usepackage{lmodern}

\usepackage[ngerman]{babel}

\usepackage{amsmath}

\usepackage{graphicx}


 

\title{Versuch E1\\ Drehspulinstrument}

\author{Frederik Strothmann, Henrik Jürgens}

\date{\today}


\begin{document}


 %deckblatt erstellen

\maketitle
\tableofcontents
\newpage


%einleitung zu dem experiment

\section{Einleitung}

Ein einfaches Drehspulinstrument soll durch Wahl geeigneter Vor- und Parallelwiderstände
zur Messung von Strömen und Spannungen verwendet werden.
Die Elektromotorische Kraft (EMK) und der Innenwiderstand sind wichtige Kenngrößen einer Spannungsquelle.

Im Rahmen dieses Versuchs sollen diese beiden Größen einer Batterie ausgemessen werden.
Nichtlineare Widerstandselemente spielen eine wichtige Rolle auf dem Gebiet der Elektronik. Ausgemessen werden soll die Strom–Spannungs–Charakteristik einer Glühlampe
und einer Halbleiterdiode.
%versuchsaufbau mit skizze

\section{Versuchsaufbau}
Der Versuchaufbau besteht aus einem Netzgerät, digitalem- und analogen Multimeter, sowie ein Versuchkasten, welcher mit fünf Widerständen, einer Diode, einer Lampe, einer Batterie und einem Drehspulinstument versehen ist. Die Komponenten haben alle vorgefertigte Kontakte zum verbinden.


\begin{figure}[htbp] 
  \centering
    \includegraphics[scale = 0.5]{versuchsaufbau.JPG}
  	\caption[Foto der für den Versuch verwendeten Geräte]{Foto der für den Versuch verwendeten Geräte\footnotemark}
  \label{fig:versuchsaufbau}
\end{figure}
\footnotetext{Graphik wurde am 13.08.2014 von der Seite: http://www.atlas.uni-wuppertal.de/~kind/apjpg/ap1e1a.JPG entnommen}

Die Aufbauten der Aufgaben werden bei der jeweiligen Durchführung angegeben.

\section{Versuchsdurchführung}


\subsection{Praktische Durchführung}

\begin{enumerate}

	\item 
	Berechnen Sie den Innenwiderstand 				eines Drehspulmeßgerätes. Das Instrument
	zeigt Vollausschlag bei einem Strom von 			1 mA bzw. einer Spannung von 200 mV.
	\item
	Bauen Sie aus diesem Instrument mit 				Hilfe der Ihnen zur Verfügung stehenden 			Widerstände $\text{R}_1$ bis 					$\text{R}_5$ ein 								Strommeßgerät mit Vollausschlag bei 300 			mA. Wie groß ist sein Innenwiderstand?
	\newpage
	\begin{figure}[htbp] 
	  \centering
	    \includegraphics[scale = 0.1]{tafel_1.JPG}
	  	\caption[Abbildung der Schaltung, der Widerstände parallel zum Drehspulinstrument]{Abbildung der Schaltung, der Widerstände}
	  \label{fig:tafel_1}
	\end{figure}
		
	
	\item
	Das soeben gebaute Strom-Meßinstrument 			soll bei der Bestimmung des unbekannten 			Widerstands $\text{R}_x$ benutzt werden. 	Der Widerstand soll durch Strom- und 			Spannungsmessung mit Hilfe von zwei 				Schaltungen bestimmt werden.
	
	\begin{figure}[htbp] 
	  \centering
	    \includegraphics[trim = 20mm 107mm 20mm 60mm, clip, scale = 1]{abb_versuch_3.pdf}
	  	\caption[Abbildung der beiden Schaltungen, für die Bestimmung des Wiederstandes]{Abbildung der beiden Schaltungen, für die Bestimmung des Wiederstandes\footnotemark}
	  \label{fig:abb_versuch_3}
	\end{figure}
	\footnotetext{Abbildung entnommen von http://www.atlas.uni-wuppertal.de/~kind/E1.pdf Seite 10 am 16.08.2014}
	
	Berücksichtigen Sie bei 	der Bestimmung 			von $\text{R}_x$ die Innenwiderstände 			der in der Schaltung vorhandenen 				Meßinstrumente. Fertigen Sie zur 				Berechnung jeweils ein Ersatzschaltbild 			an. Zur Spannungsmessung benutzen Sie 			das Digitalmultimeter (DMM) mit einem 			Innenwiderstand von 10 k$\Omega$. Als 			Spannungsquelle steht Ihnen ein 					Netzgerät zur Verfügung. Benutzen Sie 			einen der variablen Ausgänge. Stellen 			Sie zunächst -- bevor Sie die 					Schaltungen 	zusammenstecken -- mit Hilfe 	des DMM die Ausgangsspannung von ca. 			1 V ein. Der Knopf für die 						Strombegrenzung ist auf Maximum zu 				stellen.
	\item
	Bauen Sie nun mit dem Drehspulinstrument 	und den Widerständen $\text{R}_1$ bis 			$\text{R}_5$ ein Spannungsmeßgerät mit 			Vollausschlag bei 1,75 V. Wie groß ist 			sein Innenwiderstand? Verwenden Sie 				dieses Gerät zur Messung der Spannung 			der Batterie. Messen Sie dieselbe 				Spannung mit dem DMM und dem 					Vielzweckmeßgerät (Unigor). Den 					Innenwiderstand des Unigors entnehmen 			Sie einer Tabelle auf der Rückseite des 			Meßgerätes. Erklären Sie die 					unterschiedlichen Meßergebnisse.
	(Die Batterie sollte nur durch 					kurzen Druck auf den Taster belastet 			werden. Warum?)
	
\begin{figure}[htbp] 
	  \centering
	    \includegraphics[scale = 0.1]{tafel_2.JPG}
	  	\caption[Abbildung der Schaltung, der Wiederstände]{Abbildung der Schaltung, der Wiederstände}
	  \label{fig:tafel_2}
	\end{figure}	
	
	\item
	Messen Sie den Innenwiderstand der 				Batterie. Er wird größer sein als bei 			einer normalen Batterie, da ein 					”zusätzlicher Innenwiderstand“ eingebaut 	ist. Verwenden Sie zur Spannungsmessung 			das Vielzweckinstrument (Unigor). Messen 	Sie U und I und tragen Sie U gegen I 			auf. Bestimmen Sie die EMK $\text{U}_0$ und 		den Innenwiderstand $\text{R}_i$.
	
	\begin{figure}[htbp] 
	  \centering
	    \includegraphics[trim = 1mm 102mm 1mm 55mm, clip, scale = 1]{abb_versuch_5_6a.pdf}
	  	\caption[Abbildung der Schaltung, für die Bestimmung des Wiederstandes R und der Spannung U]{Abbildung der Schaltung, für die Bestimmung des Wiederstandes R\textsubscript{i} und der Spannung U\textsubscript{0}\footnotemark}
	  \label{fig:abb_versuch_5}
	\end{figure}
	\footnotetext{Abbildung entnommen von http://www.atlas.uni-wuppertal.de/~kind/E1.pdf Seite 11 am 16.08.2014}
	\newpage
	
	\item
	Messen Sie die Strom-Spannungs-					Charakteristik (Kennlinie) einer 				Glühlampe und einer Diode. Stellen Sie 			die Ergebnisse graphisch dar. Verwenden 			Sie zur Spannungsmessung das DMM und zur 	Strommessung das Unigor. Als 					Spannungsquelle benutzen Sie den unteren 	Ausgang des Netzgerätes mit einer 				Ausgangsspannung von 5 V.
	
	a) zur Messung der Kennlinie der 				Glühlampe
	
	\begin{figure}[htbp] 
	  \centering
	    \includegraphics[trim = 18mm 15mm 1mm 151mm, clip, scale = 1]{abb_versuch_5_6a.pdf}
	  	\caption[Abbildung der Schaltung, für die Bestimmung der Kennlinie der Glühlampe]{Abbildung der Schaltung, für die Bestimmung der Kennlinie der Glühlampe\footnotemark}
	  \label{fig:abb_versuch_6a}
	\end{figure}
	\footnotetext{Abbildung entnommen von http://www.atlas.uni-wuppertal.de/~kind/E1.pdf Seite 11 am 16.08.2014}
	\newpage
	b) zur Messung der Kennlinie einer Diode
	
	\begin{figure}[htbp] 
 	 \centering
 	   \includegraphics[trim = 1mm 152mm 1mm 15mm, clip, scale = 1]{abb_versuch_6b.pdf}
  	\caption[Abbildung der Schaltung, für die Bestimmung der Kennlinie der Diode]{Abbildung der Schaltung, für die Bestimmung der Kennlinie der Diode\footnotemark}
  	\label{fig:abb_versuch_6b}
	\end{figure}
	\footnotetext{Abbildung entnommen von http://www.atlas.uni-wuppertal.de/~kind/E1.pdf Seite 12 am 16.08.2014}
	Was bewirkt der Vorwiderstand
	$\text{R}_v? (\text{R}_v= 16\Omega)$
\end{enumerate}


\subsection{Theoretische Durchführung}

\begin{enumerate}

	\item
	Um den Innenwiderstand R$_i$ des 				Drehspulmeßgerätes zu bestimmen benutzen 	wir die Formel:
	\begin{align}
	\text{R}_{i}= \frac{\text{U}_0}					{\text{I}_k}
	\label{eqn:aufgabe_1}
	\end{align}
	U$_0$ die Maximalspannung, I$_k$ der maximale Klemmenstrom.
	
	Fehler:
	\begin{align}
	\sigma_{\text{R}_i} = 							\sqrt{\left(\frac{1}{\text{I}_k}\sigma_{\text{U}_0}\right)^2
	+\left(\frac{\text{U}_0}{(\text{I}_k)^2}\sigma_{\text{I}_k}\right)^2}
	\end{align}
	
	Da wir keinen Fehler auf U$_0$ und I$_k$ angenommen haben wird diese Fehlerformel nicht verwendet. 
	\item
	Für den neuen Innenwiderstand R$_{i_n}$ gilt die Formel:
	\begin{align}
	\text{R}_{i_n} = \frac{1}{\frac{1}{\text{R}_i}+\frac{\text{R}_1(\text{R}_2+
	\text{R}_3)}{\text{R}_1+\text{R}_2+\text{R}_3}}	
	\label{eqn:aufgabe_2}
	\end{align}
	Da hier jedoch gilt: R$_1$ = R$_2$ = R$_3$
	und diese mit dem selben Fehler behaftet sind vereinfacht sich die Formel zu:
	\begin{align}
	\text{R}_{i_n} = \frac{1}{\frac{1}{\text{R}_i}+\frac{2}{3\text{R}}}
	\end{align}
	mit einem Fehler von:
	\begin{align}
	\sigma_{\text{R}_{i_n}} = \sqrt{\left(\frac{2}{3\left(\frac{\text{R}}{\text{R}_i}+\frac{2}{3}\right)^2}\right)^2}\sigma_R
	\label{eqn:aufgabe_2_sigma}
	\end{align}
	%Gesamtwiderstand muss ausgerechnet werden
	\item
	Die beiden Formeln für den Widerstand 			R$_x$ sind:
	\begin{align}
	\text{R}_x=\frac{\text{U}}						{\text{I}}-\text{R}_{i_I}
	\label{eqn:aufgabe_3_schaltung_2}
	\end{align}
	\begin{align}
	\text{R}_x= 										\frac{\text{U}\text{R}_{i_U}}					{\text{I}\text{R}_{i_U}-\text{U}}
	\label{eqn:aufgabe_3_schaltung_1}
	\end{align}
	U die am Digitalmultimeter gemessene 			Spannung, I der am Drehspulinstrument 			gemessene Strom, R$_{i_I}$ der 					Innenwiderstand des Drehspulinstruments 			und R$_{i_U}$ der Innenwiderstand des DMM.
	
	Fehler:
	\begin{align}
	\sigma_{\text{R}_x}=
	\sqrt{\left(\frac{1}								{\text{I}}\sigma_U\right)^2+
	\left(\frac{U}{I^2}\sigma_I\right)^2+
	\left(\sigma_{\text{R}_{i_I}}\right)^2}
	\label{eqn:aufgabe_3_schaltung_2_sigma}
	\end{align}		
	\begin{align}
	\sigma_{\text{R}_x}=
	\sqrt{\left(\frac{\text{I}
	\text{R}_{i_U}^2}								{(\text{I}\text{R}_{i_U}-
	\text{U})^2}\sigma_U\right)^2+
	\left(\frac{\text{U}\text{R}_{i_U}^2}			{(\text{I}\text{R}_{i_U}-
	\text{U})^2}\sigma_I\right)^2+
	\left(\frac{\text{U}}							{(\text{I}\text{R}_{i_U}-
	\text{U})^2}\sigma_{
	\text{R}_{i_U}}\right)^2}
	\label{eqn:aufgabe_3_schaltung_1_sigma}
	\end{align}	 
	\item
	Der Innenwiderstand R$_{i_V}$ des Spannungsmeßgerätes berechnet sich nach der Formel:
	\begin{align}
	\text{R}_{i_V} = \text{R}_4 + \text{R}_5 + \text{R}_i 
	\label{eqn:aufgabe_4}
	\end{align}
	Mit dem Fehler:
	\begin{align}
	\sigma_{\text{R}_{i_V}} = \sqrt{\sigma_{\text{R}_4}^2 + \sigma_{\text{R}_5}^2}
	\label{eqn:aufgabe_4_sigma}
	\end{align}
	R$_i$ wurde als Fehlerfrei angenommen.
	%Während des Versuchs Dokumentieren
	
\end{enumerate}


\section{Messergebnisse}

\subsection{Grundwerte}
\begin{table}[htbp]
\caption{Werte der fünf Widerstände}
\centering
\begin{tabular}{|l|r|r|r|r|r|}
\hline
 & \multicolumn{1}{l|}{DMM} & \multicolumn{1}{l|}{} & \multicolumn{1}{l|}{Abgelesen} & \multicolumn{1}{l|}{} & \multicolumn{1}{l|}{} \\ \hline
 & \multicolumn{1}{l|}{Widerstand[Ohm]} & \multicolumn{1}{l|}{Fehler[Ohm]} & \multicolumn{1}{l|}{Widerstand[Ohm]} & \multicolumn{1}{l|}{Fehler[Ohm]} & \multicolumn{1}{l|}{Gesamtfehler[Ohm]} \\ \hline
R\_1 & 1,3 & 0,05 & 1 & 0,1 & 0,15 \\ \hline
R\_2 & 1,2 & 0,05 & 1 & 0,1 & 0,15 \\ \hline
R\_3 & 1,2 & 0,05 & 1 & 0,1 & 0,15 \\ \hline
R\_4 & 551 & 0,5 & 560 & 28 & 28 \\ \hline
R\_5 & 1000 & 0,5 & 1000 & 10 & 10 \\ \hline
\end{tabular}
\label{grundwerte}
\end{table}

\newpage

\subsection{Aufgabe 1}
\begin{table}[htbp]
\caption{Werte aus der Aufgabenstellung zu Aufgabe 1}
\centering
\begin{tabular}{|l|l|}
\hline
Strom[A] & Spannung[V] \\ \hline
\multicolumn{1}{|r|}{0,001} & \multicolumn{1}{r|}{0,2} \\ \hline
\end{tabular}
\label{aufgabe_1_grundwerte}
\end{table}


\subsection{Aufgabe 2}
\begin{table}[htbp]
\caption{Werte der Ströme in der zweiten Aufgabe}
\centering
\begin{tabular}{|l|r|}
\hline
 & \multicolumn{1}{l|}{Strom[A]} \\ \hline
I\_1 & 0,001 \\ \hline
I\_2 & 0,299 \\ \hline
\end{tabular}
\label{aufgabe_2_strom}
\end{table}

\begin{table}[htbp]
\caption{Werte der Widerstände in der zweiten Aufgabe}
\centering
\begin{tabular}{|l|r|r|}
\hline
Bezeichung & \multicolumn{1}{l|}{Widerstände[Ohm]} & \multicolumn{1}{l|}{Fehler} \\ \hline
R\_i & 200 & 0 \\ \hline
R\_m & 0,67 & 0 \\ \hline
R\_i\_n & 0,67 & 0,1 \\ \hline
\end{tabular}
\label{aufgabe_2_Widerstände}
\end{table}
R\_m ist leicht größer als R\_i und wird dazu parallel geschaltet.

\newpage

\subsection{Aufgabe 3}

\begin{table}[htbp]
\caption{Werte der Widerstände in der dritten Aufgabe}
\centering
\begin{tabular}{|l|r|}
\hline
Bezeichnung & \multicolumn{1}{l|}{Widerstände[Ohm]} \\ \hline
R\_i\_U & 10000 \\ \hline
R\_i\_I & 0,7 \\ \hline
\end{tabular}
\label{aufgabe_3_Widerstände}
\end{table}

\begin{table}[htbp]
\caption{Werte des ersten Aufbaus}
\centering
\begin{tabular}{|l|l|l|l|}
\hline
Aufbau 1 &  &  &  \\ \hline
Spannung[V] & Fehler & Strom[A] & Fehler \\ \hline
\multicolumn{1}{|r|}{0,873} & \multicolumn{1}{r|}{0,006} & \multicolumn{1}{r|}{0,174} & \multicolumn{1}{r|}{0,006} \\ \hline
Nach Aufgabenstellung &  & Mit URI &  \\ \hline
R\_x & Fehler & R\_x & Fehler \\ \hline
\multicolumn{1}{|r|}{5,0} & \multicolumn{1}{r|}{0,2} & \multicolumn{1}{r|}{5,0} & \multicolumn{1}{r|}{0} \\ \hline
\end{tabular}
\label{aufgabe_3_aufbau_1}
\end{table}

\begin{table}[htbp]
\caption{Werte des zweiten Aufbaus}
\centering
\begin{tabular}{|l|l|l|l|}
\hline
Aufbau 2 &  &  &  \\ \hline
Spannung[V] & Fehler & Strom[A] & Fehler \\ \hline
\multicolumn{1}{|r|}{0,998} & \multicolumn{1}{r|}{0,006} & \multicolumn{1}{r|}{0,174} & \multicolumn{1}{r|}{0,006} \\ \hline
Nach Aufgabenstellung &  & Mit URI &  \\ \hline
R\_x & Fehler & R\_x & Fehler \\ \hline
\multicolumn{1}{|r|}{5,1} & \multicolumn{1}{r|}{0,2} & \multicolumn{1}{r|}{5,7} & \multicolumn{1}{r|}{0} \\ \hline
\end{tabular}
\label{aufgabe_3_aufbau_2}
\end{table}

\newpage

\subsection{Aufgabe 4}

\begin{table}[htbp]
\caption{Werte der Widerstände für Aufgabe 4}
\centering
\begin{tabular}{|l|r|r|}
\hline
Bezeichnung & \multicolumn{1}{l|}{Widerstände[Ohm]} & \multicolumn{1}{l|}{Fehler} \\ \hline
R\_i & 200 & 0 \\ \hline
R\_s (R4 und R5) & 1560 & 114 \\ \hline
\end{tabular}
\label{aufgabe_4_widerstände}
\end{table}

\begin{table}[htbp]
\caption{Werte für Strom, Spannung und den Widerstand}
\centering
\begin{tabular}{|l|l|}
\hline
Strom[A] & Neue Spannung[V] \\ \hline
\multicolumn{1}{|r|}{0,001} & \multicolumn{1}{r|}{1,76} \\ \hline
R\_i\_V & Fehler \\ \hline
\multicolumn{1}{|r|}{1760} & \multicolumn{1}{r|}{30} \\ \hline
\end{tabular}
\label{aufgabe_4_werte}
\end{table}

\begin{table}[htbp]
\caption{Die gemessenen Werte}
\centering
\begin{tabular}{|l|l|l|}
\hline
Drehspulinstument &  &  \\ \hline
Spannung[V] & Fehler & Innenwiderstand[Ohm] \\ \hline
\multicolumn{1}{|r|}{0,7} & \multicolumn{1}{r|}{0,4} & \multicolumn{1}{r|}{1760} \\ \hline
DMM &  &  \\ \hline
Spannung[V] & Fehler & Innenwiderstand[Ohm] \\ \hline
\multicolumn{1}{|r|}{1,523} & \multicolumn{1}{r|}{0,006} & \multicolumn{1}{r|}{10000} \\ \hline
Unigor &  &  \\ \hline
Spannung[V] & Fehler & Innenwiderstand[Ohm] \\ \hline
\multicolumn{1}{|r|}{1,51} & \multicolumn{1}{r|}{0,02} & \multicolumn{1}{r|}{316000} \\ \hline
\end{tabular}
\label{aufgabe_4_messwerte}
\end{table}

\newpage

\subsection{Aufgabe 5}

\begin{table}[htbp]
\caption{Die gemessenen Werte für Aufgabe 5}
\centering
\begin{tabular}{|r|r|r|r|}
\hline
\multicolumn{1}{|l|}{Strom[A]} & \multicolumn{1}{l|}{Fehler} & \multicolumn{1}{l|}{Spannung[V]} & \multicolumn{1}{l|}{Fehler} \\ \hline
0,0002 & 0,00006 & 1,11 & 0,02 \\ \hline
0,00025 & 0,00006 & 1,01 & 0,02 \\ \hline
0,0003 & 0,00006 & 0,91 & 0,01 \\ \hline
0,00035 & 0,00006 & 0,81 & 0,01 \\ \hline
0,0004 & 0,00006 & 0,72 & 0,01 \\ \hline
0,0005 & 0,00006 & 0,51 & 0,01 \\ \hline
0,0006 & 0,00006 & 0,34 & 0,01 \\ \hline
\end{tabular}
\label{aufgabe_5_messwerte}
\end{table}

\newpage

\subsection{Aufgabe 6}
\begin{table}[htbp]
\caption{Die gemessenen Werte für die Glühbirne}
\centering
\begin{tabular}{|r|r|r|r|}
\hline
\multicolumn{1}{|l|}{Strom[A]} & \multicolumn{1}{l|}{Fehler} & \multicolumn{1}{l|}{Spannung[V]} & \multicolumn{1}{l|}{Fehler} \\ \hline
0,001 & 0,0006 & 0,002 & 0,005 \\ \hline
0,0011 & 0,0006 & 0,002 & 0,005 \\ \hline
0,0012 & 0,0006 & 0,002 & 0,005 \\ \hline
0,0013 & 0,0006 & 0,002 & 0,005 \\ \hline
0,0014 & 0,0006 & 0,002 & 0,005 \\ \hline
0,0015 & 0,0006 & 0,003 & 0,005 \\ \hline
0,0016 & 0,0006 & 0,003 & 0,005 \\ \hline
0,002 & 0,0006 & 0,003 & 0,005 \\ \hline
0,0025 & 0,0006 & 0,005 & 0,005 \\ \hline
0,003 & 0,0006 & 0,005 & 0,005 \\ \hline
0,0035 & 0,0006 & 0,006 & 0,005 \\ \hline
0,004 & 0,0006 & 0,007 & 0,005 \\ \hline
0,005 & 0,0006 & 0,009 & 0,005 \\ \hline
0,006 & 0,0006 & 0,012 & 0,005 \\ \hline
\end{tabular}
\label{aufgabe_6_gluebirne}
\end{table}

\begin{table}[htbp]
\caption{Die gemessenen Werte für die Diode}
\centering
\begin{tabular}{|r|r|r|r|}
\hline
\multicolumn{1}{|l|}{Strom[A]} & \multicolumn{1}{l|}{Fehler} & \multicolumn{1}{l|}{Spannung[V]} & \multicolumn{1}{l|}{Fehler} \\ \hline
1,5 & 0,02 & 0,6 & 0,006 \\ \hline
15 & 0,2 & 0,702 & 0,006 \\ \hline
3,7 & 0,02 & 0,633 & 0,006 \\ \hline
5 & 0,06 & 0,652 & 0,006 \\ \hline
7,8 & 0,08 & 0,67 & 0,006 \\ \hline
11 & 0,1 & 0,689 & 0,006 \\ \hline
23 & 0,2 & 0,72 & 0,006 \\ \hline
42 & 0,4 & 0,745 & 0,006 \\ \hline
63 & 0,6 & 0,767 & 0,006 \\ \hline
83 & 0,8 & 0,778 & 0,006 \\ \hline
37 & 0,4 & 0,735 & 0,006 \\ \hline
\end{tabular}
\label{aufgabe_6_diode}
\end{table}

\newpage


\section{Auswertung}

\subsection{Aufgabe 1}
Nach Gleichung \ref{eqn:aufgabe_1} ergibt sich ein Wert von 200$\Omega$.

\subsection{Aufgabe 2}
Der neue Innenwiderstand ergibt sich nach Gleichung \ref{eqn:aufgabe_2} und der Fehler nach Gleichung \ref{eqn:aufgabe_2_sigma}, dabei kommt ein Wert von $0.\overline{6}(\pm 0,2)\Omega$ raus.

\subsection{Aufgabe 3}
Der Widerstand sollte mit zwei verschiedenen Schaltungen bestimmt werden (siehe Abbilung \ref{fig:abb_versuch_3}).
Bei der ersten Schaltung ergab sich der Wert nach Gleichung \ref{eqn:aufgabe_3_schaltung_1} und der Fehler nach Gleichung \ref{eqn:aufgabe_3_schaltung_1_sigma}, dabei kam ein Wert von $5,02	(\pm 0,18)\Omega$ heraus.
Verwendet man nur das Ohmsche Gesetz, dann erhält man einen Wert von 5,01$\Omega$.

Bei der zweiten Schaltung ergab sich der Wert nach Gleichung \ref{eqn:aufgabe_3_schaltung_2} und der Fehler nach Gleichung \ref{eqn:aufgabe_3_schaltung_2_sigma}, dabei ergab sich ein Wert von $5,07 (\pm 0,20)\Omega$.
Verwendet man das Ohmsche Gesetz, dann erhält man einen Wert von 5,74$\Omega$.

\subsection{Aufgabe 4}
Das Drehspulinstument soll nun zu einem Spannungsmessgerät mit Vollausschlag bei 1,76 Volt umgebaut werden, der neue Innenwiederstand ergibt sich nach Gleichung \ref{eqn:aufgabe_4} und der Fehler nach Gleichung \ref{eqn:aufgabe_4_sigma}.
Dabei ergibt sich ein Wert von $1760 (\pm 30)$ Ohm.
Für die Spannung der Batterie ergab sich mit dem Drehspulinstrument ein Wert von $0,704 (\pm 0,352)$ Volt, mit dem DMM $1,523 (\pm 0,006)$ Volt, bei einem Innenwiederstand von 10000$\Omega$ und mit dem Unigor ergab sich eine Spannung von $1,51 (\pm 0,02)$ Volt, bei einem Innenwiederstand von 316k$\Omega$.
Der Unterschied entsteht durch die unterschiedlich großen Innenwiderständen.

\subsection{Aufgabe 5}
Bei der Messung der Spannung in Abhängigkeit der Stromstärke ergab sich folgender Plot:

\begin{figure}[htbp] 
	 \centering
	   \includegraphics[scale = 1]{Widerstand.pdf}
	 	\caption[Graphische Darstellung der Messung zur Bestimmung des Innerenwiderstandes der Batterie]{Graphische Darstellung der Messung zur bestimmung des Innerenwiderstandes der Batterie}
	 \label{fig:aufgabe_5_plot}
\end{figure}

Dabei ergab sich eine Steigung von $-1940 (\pm 24)\Omega$, also ist der Innenwiderstand $1940 (\pm 24)\Omega$.
Für den Y-Achsenabschnitt ergab sich ein Wert von $1,49	(\pm 0,09)$ Volt, was U$_0$ entspricht.

\newpage

\subsection{Aufgabe 6}
Im ersten Teil sollte die Kennlinie einer Glühlampe ausgemessen werden dabei ergab sich folgender Plot:

\begin{figure}[htbp] 
	 \centering
	   \includegraphics[scale = 1]{Gluebirne.pdf}
	 	\caption[Graphische Darstellung der Messung für die bestimmung der Kennline der Glühbirne]{Graphische Darstellung der Messung für die bestimmung der Kennline der Glühbirne}
	 \label{fig:aufgabe_6_a_plot}
\end{figure}

Für die Diode ergab sich der folgende Plot:

\begin{figure}[htbp] 
	 \centering
	   \includegraphics[scale = 1]{Diode.pdf}
	 	\caption[Graphische Darstellung der Messung für die Bestimmung der Kennline der Diode]{Graphische Darstellung der Messung für die Bestimmung der Kennline der Diode}
	 \label{fig:aufgabe_6_a_plot}
\end{figure}

\newpage

\section{Diskussion}

Die erste Aufgabe war sehr kurz, und da wir keine Fehlerangaben hatten, haben wir den Innenwiderstand unseres Drehspulinstruments als Fehlerlos angenommen.\\
Beim zusammenschalten des erweiterten Drehspulinstruments fiel auf, dass die Widerstände R$_1$ bis R$_5$, welche wir mit dem DMM nachgemessen haben, ca. 0.25$\Omega$ größer waren als angegeben. Dies ist ein systematischer Fehler, der z.B. durch Verunreinigungen der Kabelanschlüsse verursacht wird. Aufgabe 2 war ebenfalls nach kurzer Zeit erledigt, da lediglich der neue Innenwiderstand zu bestimmen war.\\
Unsere Messergebnisse in Aufgabe 3 liegen nach den beiden "genauen" Formeln für die verschiedenen aufbauten sehr nahe beieinander und lassen auf einen Widerstand von ca. 5$\Omega$ schließen. Die Berechnung des Widerstandes mit der Formel U = RI für den ersten Aufbau lag ebenfalls sehr nahe an 5$\Omega$, dagegen beim zweiten Aufbau weit außerhalb der Fehlerbalken, was an dem Innenwiderstand des Strommessgerätes von $\frac{2}{3}\Omega$ liegt, den man zusätzlich misst.\\
Für Aufgabe 4 musste das Drehspulinstrument in ein Spannungsmessgerät umgebaut werden. Bei der Messung der Spannung der Batterie mit den verschiedenen Spannungsmessgeräten DMM, Unigor und dem umgebauten Drehspulinstrument fiel auf, dass das Drehspulinstrument weniger als die Hälfte der mit DMM und Unigor gemessenen Spannung anzeigte, wobei der Fehler der Messung mit dem Drehspulinstrument relativ groß war. Dieses Ergebnis veranschaulicht gut, dass eine Batterie keine Ideale Spannungsquelle ist und man die EMK am besten mit hochohmigen Messgeräten misst oder an Regressionsgeraden eines U-I Plots abliest.\\
In Aufgabe 5 sollten wir mit regelbarem Widerstand U gegen I autragen, sowie U$_0$ und R$_i$ daraus bestimmen. Es ergab sich eine etwas kleinere EMK als in Aufgabe 4, was an der Ungenauigkeit unserer Regressionsgeraden liegt, die hier nach der Methode der kleinsten Fehlerquadrate angepasst wurde.\\
Der Plot der Glühlampe in Aufgabe 6 ist nicht sehr aussagekräftig, da wir mit zu niedrigen Strömen gearbeitet haben, wodurch die Glühbirne nicht leuchtete. Also zeigte sie nur das Verhalten eines linearen Widerstandes.
Am Plot der Diode kann man gut erkennen, dass der Widerstand ab einer Spannung von ca 0,8 V gegen null konvergiert. Genauere Angaben der Grenzspannung sind nicht möglich, da wir in diesem Bereich zu wenig Messwerte haben.\\
Abschließend lässt sich sagen, dass die Messungen bis auf den ersten Teil der 6. Aufgabe zufriedenstellend verlaufen sind.

 %Werte stimmen mit den Formeln überein/nicht überein

\end{document}


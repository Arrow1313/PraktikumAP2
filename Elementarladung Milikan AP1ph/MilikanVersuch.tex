\documentclass[12pt]{scrartcl}

 

\usepackage[utf8]{inputenc}

\usepackage[T1]{fontenc}

\usepackage{lmodern}

\usepackage[ngerman]{babel}

\usepackage{amsmath}

\usepackage{graphicx}


 

\title{Versuch AP1ph\\ Messung der Elementarladung - Der Milikansche Öltröpfchenversuch}

\author{Frederik Strothmann, Henrik Jürgens}

\date{\today}


\begin{document}


 %deckblatt erstellen

\maketitle
\tableofcontents
\newpage

%einleitung zu dem experiment

\section{Einleitung}

Ein lange bekanntes Quantenphänomen aus dem Bereich der Physik ist die Tatsache, dass es eine kleinste
Ladungseinheit, nämlich die Elementarladung $e$ gibt. Im Jahr 1909 gelang es R. A. MILLIKAN, diese Elementarladung direkt zu messen, indem er die Bewegung mikroskopisch kleiner Öltröpfchen im homogenen Feld eines Plattenkondensators studierte. Dieser Versuch soll wiederholt und ein möglichst genauer Wert für $e$ ermittelt werden.
%versuchsaufbau mit skizze

\section{Versuchsaufbau}


\section{Versuchsdurchführung}


\subsection{Praktische Durchführung}


\subsection{Theoretische Durchführung}


\section{Messergebnisse}



\section{Auswertung}


\section{Diskussion}

\section{Fazit}

 %Werte stimmen mit den Formeln überein/nicht überein

\end{document}


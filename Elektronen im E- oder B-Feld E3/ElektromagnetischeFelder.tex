\documentclass[12pt]{scrartcl}

 

\usepackage[utf8]{inputenc}

\usepackage[T1]{fontenc}

\usepackage{lmodern}

\usepackage[ngerman]{babel}

\usepackage{amsmath}

\usepackage{graphicx}


 

\title{Versuch E3\\ Elektronen im elektrischen und magnetischen Feld}

\author{Frederik Strothmann, Henrik Jürgens}

\date{\today}


\begin{document}


 %deckblatt erstellen

\maketitle
\tableofcontents
\newpage

%einleitung zu dem experiment

\section{Einleitung}
Mit Hilfe einer Oszillographenröhre untersuchen wir die Bewegung von Elektronen unter dem Einfluß äußerer Felder. Dazu beschäftigen wir uns zunächst mit dem Grundprinzip eines Oszillographen, der Ablenkung von Elektronenstrahlen durch elektrische Felder. Ebenso können Elektronenstrahlen durch magnetische Felder abgelenkt werden, hierzu verwenden Sie das Magnetfeld eines Helmholtz-Spulenpaares. Zum Schluß versuchen wir, mit diesem Effekt Richtung und Größe des Erdmagnetfeldes zu bestimmen.

%versuchsaufbau mit skizze

\section{Versuchsaufbau}


\section{Versuchsdurchführung}


\subsection{Praktische Durchführung}


\subsection{Theoretische Durchführung}


\section{Messergebnisse}



\section{Auswertung}


\section{Diskussion}


 %Werte stimmen mit den Formeln überein/nicht überein

\end{document}


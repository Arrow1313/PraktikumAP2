\documentclass[12pt]{scrartcl}

 

\usepackage[utf8]{inputenc}

\usepackage[T1]{fontenc}

\usepackage{lmodern}

\usepackage[ngerman]{babel}

\usepackage{amsmath}

\usepackage{graphicx}


 

\title{Versuch AP4\\ Inelastische Streuung - Das Franck-Hertz-Experiment}

\author{Frederik Strothmann, Henrik Jürgens}

\date{\today}


\begin{document}


 %deckblatt erstellen

\maketitle
\tableofcontents
\newpage

%einleitung zu dem experiment

\section{Einleitung}

Beim diesem Versuch wird ein Experiment von Franck und Hertz aus dem Jahre 1914 wiederholt. Wir werden untersuchen, daß Quecksilberatome bei inelastischen Stößen mit Elektronen Energie aufnehmen können, wenn diese dem Energieunterschied zweier Anregungsniveaus des Quecksilbers entspricht. Dieses Experiment hatte eine große historische Bedeutung, weil damit in Absorption — neben der Emission diskreter Spektrallinien — gezeigt werden konnte, dass Atome nur diskrete Energiewerte annehmen können. Auch in der heutigen Physik
spielen inelastische Streuexperimente, z.B. von Elektronen an Kernen, Protonen und Neutronen, eine wesentliche
Rolle, da sie Aufschluß über die innere Struktur der Materie vermitteln. Neben der Anregung von Hg-Atomen wird in diesem Versuch ebenfalls die Anregung von Neonatomen beobachtet.
%versuchsaufbau mit skizze

\section{Versuchsaufbau}


\section{Versuchsdurchführung}


\subsection{Praktische Durchführung}


\subsection{Theoretische Durchführung}


\section{Messergebnisse}



\section{Auswertung}


\section{Diskussion}


 %Werte stimmen mit den Formeln überein/nicht überein

\end{document}


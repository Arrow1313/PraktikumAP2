\documentclass[12pt]{scrartcl}

 

\usepackage[utf8]{inputenc}

\usepackage[T1]{fontenc}

\usepackage{lmodern}

\usepackage[ngerman]{babel}

\usepackage{amsmath}

\usepackage{graphicx}


 

\title{Versuch E2\\ Der Halleffekt}

\author{Frederik Strothmann, Henrik Jürgens}

\date{\today}


\begin{document}


 %deckblatt erstellen

\maketitle
\tableofcontents
\newpage

%einleitung zu dem experiment

\section{Einleitung}

In diesem Versuch sollen Sie das Magnetfeld einer kurzen Spule auf ihrer Mittelachse ausmessen und dabei einfache elektrische Grundschaltungen anwenden. Zur Messung benutzen Sie eine Hallsonde, die zu diesem Zweck vorher geeicht werden muß. Die Hallsonde liefert eine Spannung, die proportional zum Magnetfeld ist, aber schwierig zu messen ist,
da sie im Millivoltbereich liegt und einige Hallsondentypen eine hohen Innenwiderstand
haben. Die Ausgangsspannung der Hallsonde wird daher in diesem Versuch mit einer
Kompensationsschaltung bestimmt. Den Widerstand der Hallsonde messen Sie mit der
Wheatstoneschen Brückenschaltung.

%versuchsaufbau mit skizze

\section{Versuchsaufbau}


\section{Versuchsdurchführung}


\subsection{Praktische Durchführung}


\subsection{Theoretische Durchführung}


\section{Messergebnisse}



\section{Auswertung}


\section{Diskussion}


 %Werte stimmen mit den Formeln überein/nicht überein

\end{document}


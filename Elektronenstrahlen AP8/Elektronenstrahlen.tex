\documentclass[12pt]{scrartcl}

 

\usepackage[utf8]{inputenc}

\usepackage[T1]{fontenc}

\usepackage{lmodern}

\usepackage[ngerman]{babel}

\usepackage{amsmath}

\usepackage{graphicx}


 

\title{Versuch AP8\\ Elektronenstrahlen}

\author{Frederik Strothmann, Henrik Jürgens}

\date{\today}


\begin{document}


 %deckblatt erstellen

\maketitle
\tableofcontents
\newpage

%einleitung zu dem experiment

\section{Einleitung}

Die Entdeckung der Welleneigenschaften von massebehafteten Teilchen war historisch von größter Bedeutung
für die Entwicklung der Quantenmechanik. Dabei nimmt die Schrödingergleichung als charakteristische Wellengleichung einen besonderen Stellenwert ein, da sie die Grundlage für die nicht-relativistische Quantenmechanik
bildet. In diesem Versuch führen wir zwei Teilexperimente durch, die die Teilchen- bzw. Wellennatur von Elektronen demonstrieren. Mit Hilfe eines Fadenstrahlrohres bestimmen wir die spezifische Ladung e/m eines Elektrons, das im homogenen Magnetfeld eines Helmholtzspulenpaares auf eine Kreisbahn gezwungen wird.
Die Welleneigenschaften eines Elektronenstrahles wird durch Beugung von Elektronen an einer dünnen Schicht von Kohlenstoffkristallen untersucht. Mit Hilfe der bekannten Gitterkonstanten dieser Kristalle soll
die Gültigkeit der DeBroglie-Beziehung nachgewiesen werden. Dabei sollen die Grenzen nicht-relativistischer
Beschreibungsweise sowie der Einfluß von Wechselwirkungen zwischen den Elektronen und dem inneren Potential der Kohlenstoffkristalle berücksichtigt werden.
%versuchsaufbau mit skizze

\section{Versuchsaufbau}


\section{Versuchsdurchführung}


\subsection{Praktische Durchführung}


\subsection{Theoretische Durchführung}


\section{Messergebnisse}



\section{Auswertung}


\section{Diskussion}


 %Werte stimmen mit den Formeln überein/nicht überein

\end{document}


\documentclass[12pt]{scrartcl}

 

\usepackage[utf8]{inputenc}

\usepackage[T1]{fontenc}

\usepackage{lmodern}

\usepackage[ngerman]{babel}

\usepackage{amsmath}

\usepackage{graphicx}


 

\title{Versuch WO3\\ Beugung und Interferenz von Lichtwellen}

\author{Frederik Strothmann, Henrik Jürgens}

\date{\today}


\begin{document}


 %deckblatt erstellen

\maketitle
\tableofcontents
\newpage

%einleitung zu dem experiment

\section{Einleitung}

In der Versuchsreihe WO3 soll die Beugung (Diffraktion) von kohärenten Lichtwellen (Laserlicht) an verschiedenen Öffnungen untersucht werden.
Die Erscheinungen von Beugung und Interferenz sind grundlegend für die Wirkungsweise wichtiger optischer Instrumente. So bestimmt z.B. die Beugung an in der Regel kreisförmigen Öffnungen (z.B. Linsenfassungen) das räumliche Auflösungsvermögen aller optischer Instrumente vom Mikroskop über das menschliche Auge bis hin zu den großen Radioteleskopen. Wir werden zunächst die Intensitätsverteilung des Beugungsmusters eines Einzelspaltes ausmessen und unsere Messungen mit den entsprechenden Rechnungen, die auf dem Huygensschen Prinzip beruhen, vergleichen. Das gleiche werden wir dann anschließend mit den Beugungsbildern einer Lochblende, eines Doppelspaltes und eines Strichgitters durchfÜhren.
%versuchsaufbau mit skizze

\section{Versuchsaufbau}


\section{Versuchsdurchführung}


\subsection{Praktische Durchführung}


\subsection{Theoretische Durchführung}


\section{Messergebnisse}



\section{Auswertung}


\section{Diskussion}


 %Werte stimmen mit den Formeln überein/nicht überein

\end{document}


\documentclass[12pt]{scrartcl}

 

\usepackage[utf8]{inputenc}

\usepackage[T1]{fontenc}

\usepackage{lmodern}

\usepackage[ngerman]{babel}

\usepackage{amsmath}

\usepackage{graphicx}


 

\title{Versuch AP2ph\\ Bestimmung des Planckschen Wirkungsquantums - Der Photoelektrische Effekt}

\author{Frederik Strothmann, Henrik Jürgens}

\date{\today}


\begin{document}


 %deckblatt erstellen

\maketitle
\tableofcontents
\newpage

%einleitung zu dem experiment

\section{Einleitung}

In diesem Versuch wird eine Photozelle (mit Meßverstärker) mit verschiedenen Spektrallinien einer Quecksilberdampflampe beleuchtet, um einen Vergleich zwischen klassischer Theorie und Quantentheorie durchzuführen.
%1
Außerdem wird das Verhältnis $\frac{h}{e}$ und hieraus die Plancksche Konstante h
bestimmt.
%2
In ergänzenden Versuchsteilen können dann umgekehrt aus der Stoppspannung die Wellenlänge des Lichts verschiedenfarbiger Leuchtdioden bestimmt werden.
%3
Außerdem schätzen wir die Quantenausbeute der Photozelle und die Lichtleistung der Quecksilber-Spektrallinien
unter Verwendung einer Halbleiter-Photodiode ab.
%versuchsaufbau mit skizze

\section{Versuchsaufbau}

\section{Vergleich zwischen Wellen- und Quantenmodell und Bestimmung der Planckschen
Konstante $h$}
\subsection{Versuchsdurchführung}


\subsubsection{Praktische Durchführung}
Das Quantenmodell des Lichts besagt, dass die maximale Energie $E_{kin,max}$ von Photoelektronen nur von der Frequenz des einfallenden Lichts abhängt und von der Lichtintensität unabhängig ist.
Dagegen sagt das klassische Wellenmodell des Lichts voraus, dass die maximale Energie $E_{kin,max}$ der Photoelektronen von der Lichtintensität abhängt.
Entsprechend dem Quantenmodell des Lichts ist die Energie der Lichtquanten direkt proportional zu ihrer Frequenz: Je höher die Frequenz, desto mehr Energie hat das Quant. Beim Experimentieren wollen wir den Proportionalitätsfaktor, die Plancksche Konstante, bestimmen.
Wir werden dazu unterschiedliche Spektrallinien der Quecksilberdampflampe verwenden und die Maximalenergie $E_{kin,max}$ der Photoelektronen als Funktion der Wellenlänge und Frequenz untersuchen:
\begin{enumerate}
\item Im (Gitter-)Spektrum der Quecksilberdampflampe sehen wir 5 Farben in mindestens 2 Ordnungen.
%Abbildung 5 aus der versuchsbeschreibung mit den ordnungen einfügen
Wir stellen nun den h/e-Apparat so ein, daß nur eine Farbe aus dem Spektrum erster Ordung (die zweithellste Ordnung) in die Offnung der Photozelle fällt.
\item Wir messen für jede Farbe der ersten Ordnung die Stoppspannung mit dem Digitalvoltmeter und setzen den
Gelb- bzw. Grünfilter vor den weißen Schirm, falls wir die gelbe bzw. grüne Spektrallinie vermessen.
\item Danach wiederholen wir die Messung für die fünf Farben zweiter Ordnung.
\item Wir benutzen dann die Transmissionsfilter (Graufilter), um die Intensitätsabhängigkeit der Stoppspannung zu messen.
\end{enumerate}
\subsubsection{Verwendete Formeln}
\subsection{Messergebnisse}
\subsection{Auswertung}
%Du musst bei dieser aufgabe beachten, was in der auswertung steht!
\subsection{Diskussion}

\section{Abschätzung des Photostroms; Quantenausbeute der Photozelle und Lichtleistung}
\subsection{Versuchsdurchführung}
\subsubsection{Praktische Durchführung}
\paragraph{1. Kondensatorkapazität und Photostrom}
Die Photozelle und der Eingang des Meßverstärkers im h/e-Apparat bilden einen kleinen Kondensator, der durch den Photostrom aufgeladen und durch Leckströme im Meßverstärker entladen wird.
Wir wollen aus der Halbwertszeit für den Auf- und Entladevorgang, sowie der Stoppspannung $U_0$ die Kondensatorkapazität und den Photostrom abschätzen. Die Entladekurven haben wir mit einem Oszilloskop vermessen, da unsere Leckströme sehr groß waren.

\paragraph{2. Die Photozelle als Spektrometer}
Es stehen uns verschiedene Leuchtdioden zur Verfügung. Diese schließen wir an ein Netzgerät an.
(Strom etwa 20 mA, wird durch einen Widerstand bzw. Regler im Anschlußstecker begrenzt), Es ist zu beachten, dass Leuchtdioden -- wie gewöhnliche Dioden auch -- nur in einer Stromrichtung arbeiten.
Wir befestigen die Leuchtdioden nacheinander in der 5-mm-Bohrung einer Kunststoffhalterung und schieben die Halterung waagerecht so über den weißen Schirm des h/e-Apparates, dass das Licht auf die Photozelle gelangen kann. Wir bestimmen dann die jeweiligen Stoppspannungen.
Mit den Ergebnissen aus Versuchsteil 1 können wir, durch Messung der Stoppspannung, die Wellenlänge des Lichtes bestimmen. Diese vergleichen wir mit den Wellenlängen, welche in den Datenblättern der Leuchtdioden angegeben sind. (siehe auch Beschriftung an den Leuchtdioden):
rot = 645 nm, gelb = 583 nm, grün = 565 nm, blau = 470 nm.
%Erklären Sie die Abweichungen zum Literaturwert.
\paragraph{3. Quantenausbeute der Photozelle und Lichtleistung}
Nicht jedes erzeugte Photoelektron gelangt zur Anode, wenn -- wie in unserem Versuchsaufbau -- die
Photozelle ohne hohe Vorspannung betrieben wird. Die Zahl der im Photostrom nachgewiesenen Photoelektronen ist viel kleiner als die Anzahl der einfallenden Lichtquanten. %(warum?)
Wir wollen das Verhältnis zwischen der Anzahl $N_{ph}$ der Photonen (Lichtquanten), die von der Lichtquelle ausgehend in die Photozelle einfallen, zur
Anzahl $N_{e,Zelle}$ der Elektronen, die von diesen Photonen in der Photozelle produziert werden und den Photostrom bewirken, abschätzen, also: Quantenausbeute = $\frac{N_{e,Zelle}} {N_{ph}}$.
Dazu verwenden wir folgendes Verfahren: Für diese Messung steht uns eine Halbleiter-Photodiode zur Verfügung. Eine solche Photodiode besteht aus einem kleinen Halbleiterkristall. Fällt Licht auf die strahlungsempfindliche Fläche (hier 2, 75 $\times$ 2, 75 mm groß), so werden durch den inneren Photoeffekt Ladungsträger erzeugt, es fließt ebenfalls ein Photostrom. In der Photodiode erzeugt fast jedes einfallende Photon ein Elektron: bei einer Wellenlänge von 850 nm werden
etwa 88 Elektronen pro 100 Lichtquanten erzeugt, für andere Wellenlängen ist das Verhältnis geringer.
Die folgende Abbildung zeigt die Abhängigkeit der relativen spektralen Empfindlichkeit für unseren Photodioden-
typ BPW34 (100 \% entsprechen 88 Elektronen pro 100 Quanten).
%bitte abbildung für die relative empfindlichkeit der photodiode einfügen
Wir schließen die Photodiode an ein Digitalmultimeter an, Messbereich 200 $\mu$A. Die Polarität ist nicht wichtig. Wir halten die Photodiode in den Strahlengang, sodass das Licht einer Spektrallinie auf die Photodiode fällt. Es ist zu beachten, dass die Photodiode etwa den gleichen Abstand von der Lichtquelle hat wie die Photozelle.
%(warum?)
Wir Klappen daher das Lichtschutzrohr zur Seite
und halten die Photodiode vor der Photozelle in das Licht der Spektrallinie.
Wir messen den Photostrom für diese und die anderen Spektrallinien.
Wir können jetzt vom Photostrom der Photodiode auf die Zahl der Elektronen und daraus -- je nach Wellenlänge -- auf die Anzahl der Photonen (pro Zeitintervall) schließen. Wir nehmen an, dass die Öffnung an der Photozelle etwa so groß ist wie die Fläche der Photodiode, d.h. beide erhalten etwa gleich viel Photonen — vorausgesetzt, beide haben den gleichen Abstand zur Lichtquelle. Den Photostrom der Photozelle haben wir in Teil 1 abgeschätzt. Hiermit bestimmen wir die Quantenausbeute der Photozelle. Wir haben die Anzahl der Photonen (pro Zeitintervall) bestimmt; aus der Wellenlänge und der Planckschen Konstanten können wir die Energie jedes Photons berechnen. Wir bestimmen wir die Lichtleistung, die mit den einzelnen Spektrallinien auf die Photozelle gelangt.

\subsubsection{Verwendete Formeln}
\subsection{Messergebnisse}
\subsection{Auswertung}
\subsection{Diskussion}

\section{Fazit}

 %Werte stimmen mit den Formeln überein/nicht überein

\end{document}


\documentclass[12pt]{scrartcl}

 

\usepackage[utf8]{inputenc}

\usepackage[T1]{fontenc}

\usepackage{lmodern}

\usepackage[ngerman]{babel}

\usepackage{amsmath}

\usepackage{graphicx}


 

\title{Versuch AP2ph\\ Bestimmung des Planckschen Wirkungsquantums - Der Photoelektrische Effekt}

\author{Frederik Strothmann, Henrik Jürgens}

\date{\today}


\begin{document}


 %deckblatt erstellen

\maketitle
\tableofcontents
\newpage

%einleitung zu dem experiment

\section{Einleitung}

In diesem Versuch wird eine Photozelle (mit Meßverstärker) mit verschiedenen Spektrallinien einer Quecksilberdampflampe beleuchtet, um einen Vergleich zwischen klassischer Theorie und Quantentheorie durchzuführen. Außerdem wird das Verhältnis h/e und hieraus die Plancksche Konstante h
bestimmt. In ergänzenden Versuchsteilen können dann umgekehrt aus der Stoppspannung die Wellenlänge des Lichts
verschiedenfarbiger Leuchtdioden bestimmt werden.
Außerdem schätzen wir die Quantenausbeute der Photozelle und die Lichtleistung der Quecksilber-Spektrallinien
unter Verwendung einer Halbleiter-Photodiode ab.
%versuchsaufbau mit skizze

\section{Versuchsaufbau}


\section{Versuchsdurchführung}


\subsection{Praktische Durchführung}


\subsection{Theoretische Durchführung}


\section{Messergebnisse}



\section{Auswertung}


\section{Diskussion}


 %Werte stimmen mit den Formeln überein/nicht überein

\end{document}


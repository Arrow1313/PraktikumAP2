\documentclass[12pt]{scrartcl}

 

\usepackage[utf8]{inputenc}

\usepackage[T1]{fontenc}

\usepackage{lmodern}

\usepackage[ngerman]{babel}

\usepackage{amsmath}

\usepackage{graphicx}


 

\title{Versuch AK1\\ Ultraschall}

\author{Frederik Strothmann, Henrik Jürgens}

\date{\today}


\begin{document}


 %deckblatt erstellen

\maketitle
\tableofcontents
\newpage

%einleitung zu dem experiment

\section{Einleitung}
Mit einem System aus piezoelektrischen Ultraschallsendern und -empfängern wird die Schallgeschwindigkeit in Luft
bestimmt nach der Phasen- und der Laufzeitmethode. Bei Kenntnis der Schallgeschwindigkeit kann die Laufzeitmethode zur Entfernungsmessung eingesetzt werden (Echolotprinzip). Eine Anordnung mit zwei gleichen Ultraschallsendern wirkt wie ein Doppelspalt und liefert ein typisches
Ultraschall-Interferenzmuster, das ausgemessen werden soll.
Außerdem sollen noch die elektrischen Eigenschaften des Piezosenders (Impedanz) untersucht werden.

%versuchsaufbau mit skizze

\section{Versuchsaufbau}


\section{Versuchsdurchführung}


\subsection{Praktische Durchführung}


\subsection{Theoretische Durchführung}


\section{Messergebnisse}



\section{Auswertung}


\section{Diskussion}


 %Werte stimmen mit den Formeln überein/nicht überein

\end{document}

